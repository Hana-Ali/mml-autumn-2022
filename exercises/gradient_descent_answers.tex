\section{Answers Lecture 5: Gradient Descent Convergence}

\paragraph{Question \ref{q:rayleigh_quotient}}

As $\BA$ is symmetric, we can write the eigen-decomposition formula as $\BA = \BQ \Lambda \BQ^{\top}$ with $\BQ$ containing an orthonormal basis of $\mathbf{R}^{d\times 1}$. 
Then using the fact that $\BQ \BQ^\top = \mathbf{I}$, we can define $\z = \BQ^\top \x$ and rewrite the Rayleigh quotient as:
\begin{equation}
R(\BA, \x) = \frac{\x^\top \BQ \Lambda \BQ^\top \x}{\x^\top \BQ \BQ^\top \x} = \frac{\z^\top \Lambda \z}{\z^\top \z}.
\end{equation}
As $\Lambda = \text{diag}(\lambda_1, ..., \lambda_d)$ is a diagonal matrix, we have (writing $\z = (z_1, ..., z_d)^\top$)
\begin{equation}
    \z^\top \Lambda \z = \sum_{i=1}^d \lambda_i z_i^2.
\end{equation}
Therefore the Rayleigh quotient can be written as the following weighted average of the eigenvalues
\begin{equation}
    R(\BA, \x) = \sum_{i=1}^d \frac{z_i^2}{|| \z ||^2_2} \lambda_i, \quad \text{with } \sum_{i=1}^d \frac{z_i^2}{|| \z ||^2_2} = 1.
\end{equation}
In summary, these derivations indicate that the Rayleigh quotient is bounded as
\begin{equation}
\begin{aligned}
    &\lambda_{min}(\BA) \leq R(\BA, \x) \leq \lambda_{max}(\BA) \\
\Rightarrow \quad &\lambda_{min}(\BA) || \x ||_2^2 \leq \x^\top \BA \x \leq \lambda_{max}(\BA) || \x ||_2^2,
\end{aligned}
\end{equation}
where $\lambda_{min}(\BA)$ and $\lambda_{max}(\BA)$ are the smallest and largest eigenvalues of $\BA$, respectively.


\paragraph{Question \ref{q:pre_conditioned_gd}}
For linear regression we have $\nabla_{\mparam} L(\mparam) = \X^\top(\X \mparam_t - \y)$. Therefore in this case the pre-conditioned gradient descent update rule is
$$\mparam_{t+1} =  \mparam_t - \frac{\gamma}{\sigma^2} \BP^{-1} \X^\top(\X \mparam_t - \y).$$
%
Under the formula of Arithmetico-geometric sequence, we have
$$\mparam_{t} = (\mathbf{I} - \frac{\gamma}{\sigma^2} \BP^{-1} \X^\top \X)^t (\mparam_0 - \mparam^*) + \mparam^*, \quad \mparam^* = (\X^\top \X)^{-1} \X^\top \y.
$$
So this means if pre-conditioned gradient descent converges, it will converge to the right answer $\mparam^*$.

Now we need to show that pre-conditioned gradient descent converges with the right choices of $\gamma$ and $\BP$. This requires us to analyse the eigenvalues of the matrix $(\mathbf{I} - \frac{\gamma}{\sigma^2} \BP^{-1} \X^\top \X)^2$, and for a given $\BP$:
$$ \lambda \text{ is an eigenvalue of } \BP^{-1} \X^\top \X \quad \Rightarrow \quad (1 - \frac{\gamma}{\sigma^2})^2 \text{ is an eigenvalue of } (\mathbf{I} - \frac{\gamma}{\sigma^2} \BP^{-1} \X^\top \X)^2.$$
Following the same idea as to prove convergence for gradient descent, we have the learning rates bounds
$$ \gamma_{min} = 2 \sigma^2 / \lambda_{max}(\BP^{-1} \X^\top \X), \quad \gamma_{max} = 2 \sigma^2 / \lambda_{min}(\BP^{-1} \X^\top \X) .$$
Now to set $\gamma_{min} = \gamma_{max}$, we should choose $\BP$ such that $\lambda_{min}(\BP^{-1} \X^\top \X) = \lambda_{max}(\BP^{-1} \X^\top \X)$, and an easy way to do so is to choose $\BP \propto \X^\top \X$.

One can show that for linear regression problems, the Hessian matrix of $L(\mparam)$ is $\nabla^2_{\mparam} L(\mparam) \propto \X^\top \X$. In general for a given loss function $L(\mparam)$ we often set $\BP_t = \nabla^2_{\mparam_t} L(\mparam_t)$ if it can be computed in a fast way.


\paragraph{Question \ref{q:momemtum_gd}}
First note that $\nabla_{\mparam}L(\mparam_t) = \frac{1}{\sigma^2} \X^\top (\X \mparam_t - \y)$
The update equations for both the parameter and the momentum are
\begin{equation}
\begin{aligned}
    \mparam_{t+1} &= \mparam_t - \gamma \nabla_{\mparam} L(\mparam_t) + \alpha \Delta \mparam_t = \mparam_t - \frac{\gamma}{\sigma^2} \X^\top (\X \mparam_t - \y) + \alpha \Delta \mparam_t \\
    \Delta \mparam_{t+1} &= \mparam_{t+1} - \mparam_t =  \alpha \Delta \mparam_t - \frac{\gamma}{\sigma^2} \X^\top (\X \mparam_t - \y).
\end{aligned}
\end{equation}
Now collecting both equations together into a ``joint'' linear equation:
\begin{equation}
\begin{bmatrix}
\mparam_{t+1} \\ \Delta \mparam_{t+1} 
\end{bmatrix} =
\begin{bmatrix}
\mathbf{I} - \frac{\gamma}{\sigma^2} \X^\top \X & \alpha \mathbf{I} \\ - \frac{\gamma}{\sigma^2} \X^\top \X & \alpha \mathbf{I} 
\end{bmatrix} 
\begin{bmatrix}
\mparam_t \\ \Delta \mparam_t 
\end{bmatrix} + 
\begin{bmatrix}
\frac{\gamma}{\sigma^2} \X^\top \y \\ \frac{\gamma}{\sigma^2} \X^\top \y 
\end{bmatrix}.
\end{equation}
Then we can apply the derivation of arithmetico–geometric sequences again, and show that
\begin{equation}
\begin{bmatrix}
\mparam_{t} \\ \Delta \mparam_{t} 
\end{bmatrix} =
\begin{bmatrix}
\mathbf{I} - \frac{\gamma}{\sigma^2} \X^\top \X & \alpha \mathbf{I} \\ - \frac{\gamma}{\sigma^2} \X^\top \X & \alpha \mathbf{I} 
\end{bmatrix}^t 
\begin{bmatrix}
\mparam_0 - \mparam^* \\ \Delta \mparam_0 
\end{bmatrix} + 
\begin{bmatrix}
\mparam^* \\ \bm{0} 
\end{bmatrix},
\end{equation}
with $\mparam^* = (\X^\top \X)^{-1} \X^\top \y$. This equation also says if momentum GD converges, the momentum $\Delta \mparam_t$ will vanish to $\bm{0}$, which is as expected as $\Delta \mparam_{t} = \mparam_{t} - \mparam_{t-1} \rightarrow \bm{0}$.