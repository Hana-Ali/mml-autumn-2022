\section{Answers Lecture 1: Probability, Vectors, Differentiation}

\paragraph{\questionref{q:vecnot}} 

Given $p(\textbf{x}) = \frac{1}{C}(x_1^2 + x_1x_2 x_2^2 + 2x_2x_3)$, we need to rearrange the terms to find an expression as follows:
\begin{equation}
p(\textbf{x}) = \frac{1}{C}(\textbf{x}^T A \textbf{x}), \quad A\in \mathbb{R}^{3\times 3}
\end{equation}
Inspection of the terms in $p(\textbf{x})$ gives the following solution
\begin{align*}
\begin{pmatrix}
x_1 & x_2 & x_3
\end{pmatrix}^T
\begin{pmatrix}
1 & \frac{1}{2} & 0\\
\frac{1}{2} & 1 & 1\\
0 & 1 & 0
\end{pmatrix}
\begin{pmatrix}
x_1 & x_2 & x_3
\end{pmatrix} =
\begin{pmatrix}
x_1 + \frac{x_2}{2}\\
\frac{x_1}{2} + x_2 + x_3\\
x_2
\end{pmatrix}
\begin{pmatrix}
x_1 & x_2 & x_3
\end{pmatrix} = x_1^2 + x_1x_2 x_2^2 + 2x_2x_3
\end{align*}.
Thus,
\begin{equation}
p(\textbf{x}) = \frac{1}{C}(\textbf{x}^T A \textbf{x}), \quad A = \begin{pmatrix}
1 & \frac{1}{2} & 0\\
\frac{1}{2} & 1 & 1\\
0 & 1 & 0
\end{pmatrix}
\end{equation}


