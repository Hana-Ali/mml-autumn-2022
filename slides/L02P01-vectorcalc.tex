%% Time-stamp: <2018-10-18 20:24:12 (marc)>
\documentclass[xcolor=x11names,compress,mathserif]{beamer}

\newcommand{\hackspace}{\hspace{4.2mm}}
\newcommand{\showstudent}[1]{}
\newcommand\hmmax{0}
\newcommand\bmmax{0}


\usepackage{../includes/MarkMathCmds}





% talk/author information
\newcommand{\authorname}{Mark van der Wilk}
\newcommand{\authoremail}{m.vdwilk@imperial.ac.uk}
\newcommand{\authoraffiliation}{
  Department of Computing\\Imperial
  College London}
\newcommand{\authortwitter}{markvanderwilk}
\newcommand{\slidesettitle}{\imperialBlue{Vector Calculus}}
\newcommand{\footertitle}{Differentiation}
\newcommand{\location}{Imperial College London}
\newcommand{\talkDate}{October 12, 2021}



\date{\imperialGray{\talkDate}}




% load defaults
\selectcolormodel{rgb}
\usepackage{ifxetex,ifluatex}
\newif\ifxetexorluatex
\ifxetex
  \xetexorluatextrue
\else
  \ifluatex
    \xetexorluatextrue
  \else
    \xetexorluatexfalse
  \fi
\fi

\usepackage{textpos}
%\usepackage{arabtex}
\usepackage{tikz}
\usetikzlibrary{decorations.markings}
\usetikzlibrary{arrows}
\usetikzlibrary{shapes}
\usetikzlibrary{plotmarks}
\usetikzlibrary{mindmap,trees,backgrounds}

\tikzstyle{every picture}+=[remember picture]

%\usepackage{movie15}
% \usepackage{pdfpages}
%\usepackage{xmpmulti}

\usepackage{anyfontsize}
\usepackage{wrapfig}
\usepackage{animate}
\usepackage{multirow}
\usepackage{multimedia}
\usepackage{xmpmulti}
%\usepackage[latin9]{inputenc}
\usepackage[english]{babel}
\usepackage{scalefnt}
\usepackage{verbatim}
\usepackage{url}
% \usepackage{pgf,pgfarrows,pgfnodes}
\usepackage{textpos}
\usepackage[tight,ugly]{units}
\usepackage{url}
\usepackage{bbm}
\usepackage[english]{babel}
\usepackage{fancyhdr}
\usepackage{bm} % correct bold symbols, like \bm
\usepackage{amsmath}
\usepackage{amsfonts}
\usepackage{amssymb}
\usepackage{mathrsfs}
\usepackage{mathtools}
\usepackage{color}
\usepackage{cancel}
\usepackage{algorithm}
\usepackage{algpseudocode}
\usepackage{mathrsfs}
\usepackage{listings}
\usepackage{graphicx} % for pdf, bitmapped graphics files
\usepackage{mathtools}
\usepackage{units}
\usepackage{subfig}
\usepackage{enumerate}
\usepackage[comma,authoryear]{natbib}
\usepackage{dsfont}


\ifxetexorluatex
\usepackage{fontspec}
\setmainfont[Scale=0.8]{OpenDyslexic-Regular}
\else
\usefonttheme{professionalfonts}
\fi

\renewcommand{\vec}[1]{{\boldsymbol{{#1}}}} % vector
\newcommand{\mat}[1]{{\boldsymbol{{#1}}}} % matrix
% \newcommand{\KL}[2]{\mathrm{KL}(#1\|#2)} % KL divergence
\newcommand{\R}[0]{\mathds{R}} % real numbers
\newcommand{\Z}[0]{\mathds{Z}} % integers
\newcommand{\tr}[0]{\text{tr}} % trace
% \newcommand{\inv}{^{-1}}
% \DeclareMathOperator*{\diag}{diag}
\newcommand{\E}{\mathds{E}} % expectation
\newcommand{\var}{\mathds{V}}
\newcommand{\gauss}[2]{\mathcal{N}\big(#1,\,#2\big)}
\newcommand{\gaussx}[3]{\mathcal{N}\big(#1\,|\,#2,\,#3\big)}
\newcommand{\gaussBig}[2]{\mathcal{N}\left(#1,\,#2\right)}
\newcommand{\gaussxBig}[3]{\mathcal{N}\left(#1\,\left|\,#2,\,#3\right.\right)}
\newcommand{\Ber}[0]{\mathrm{Ber}} % Bernoulli distribution
\DeclareMathOperator{\cov}{Cov}
\ifxetexorluatex
\renewcommand{\T}[0]{^\top}
\renewcommand{\d}[0]{\text{d}} % derivative
\else
\newcommand{\T}[0]{^\top}
\renewcommand{\d}[0]{\text{d}} % derivative
\fi
% calculus
\newcommand{\pdiff}[1]{\frac{\partial}{\partial #1}}
\newcommand{\pdiffF}[2]{\frac{\partial #1}{\partial #2}}
\newcommand{\diffF}[2]{\frac{{\d}#1}{{\d}#2}}
\newcommand{\diffFII}[2]{\frac{{\d}^2 #1}{{\d}#2^2}}
\newcommand{\diff}[1]{\frac{{\d}}{{\d}#1}}
\newcommand{\diffII}[1]{\frac{{\d}^2}{{\d}#1^2}}
\newcommand{\class}[0]{\mathcal{C}}

\newcommand{\idx}[1]{{(#1)}}
% \newcommand{\norm}[1]{\left\|#1\right\|}
\newcommand{\proj}[1]{\tilde{#1}}
\newcommand{\pcacoord}{z}
\newcommand{\pcacoordnew}{\zeta}
\newcommand{\latent}{z}
% \newcommand{\given}{\,|\,}
\newcommand{\genset}[1]{\mathrm{span}[#1]} % generating set
\newcommand{\set}[1]{\mathcal{#1}} % set
\newcommand{\fixgmfont}[1]{\scalebox{0.8}{#1}}



\usepackage{pifont}% http://ctan.org/pkg/pifont
\newcommand{\cmark}{{\color{green!40!black}\ding{51}}}%
\newcommand{\xmark}{{\color{red}\ding{55}}}%
\newcommand{\green}[1]{{\bf{\textcolor{green}{#1}}}}
\newcommand{\red}[1]{{\bf{\textcolor{red}{#1}}}}

\newcommand<>\red[1]{{\color#2[rgb]{1,0,0}#1}}
\newcommand<>\blue[1]{{\color#2[rgb]{0,0,1}#1}}
\newcommand<>\yellow[1]{{\color#2{camyellow}#1}}
\newcommand<>\green[1]{{\color#2[rgb]{0,0.6,0.0}#1}}
\newcommand<>\violet[1]{{\color#2[rgb]{0.6,0,0.6}#1}}
\newcommand<>\orange[1]{{\color#2[rgb]{1,0.5,0}#1}}
\newcommand<>\black[1]{{\color#2[rgb]{0,0,0}#1}}
\newcommand<>\steel[1]{{\color#2[rgb]{0,0,0.8}#1}}
\newcommand<>\darkblue[1]{{\color#2[rgb]{0,0,0.6}#1}}
\newcommand<>\lightblue[1]{{\color#2[rgb]{0.4,0.4,0.7}#1}}
\newcommand<>\gray[1]{{\color#2[rgb]{0.4,0.4,0.4}#1}}
\newcommand<>\greenish[1]{{\color#2[rgb]{0.45, 0.66, 0.45}#1}}
\newcommand<>\redish[1]{{\color#2[rgb]{0.7843    0.3706    0.3706}#1}}
\definecolor{redishTIKZ}{rgb}{0.7843, 0.3706, 0.3706}
\definecolor{imperialBlue}{rgb}{0.058, 0.219, 0.418}
\definecolor{aimsbrown}{rgb}{0.539, 0.117, 0.015}
% \definecolor{imperialGray}{rgb}{0.414, 0.488, 0.671 }
\definecolor{imperialGray}{RGB}{109,153, 204}
\definecolor{aimslightbrown}{RGB}{138,88,84}
\newcommand<>\imperialBlue[1]{{\color#2[rgb]{0.058, 0.219, 0.418}#1}}
\newcommand<>\aimsbrown[1]{{\color#2[rgb]{0.539, 0.117, 0.015}#1}}
%\newcommand<>\imperialGray[1]{{\color#2[rgb]{0.414, 0.488, 0.671}#1}}
\newcommand<>\imperialGray[1]{{\color#2[RGB]{109,153, 204}#1}}
\newcommand<>\aimslightbrown[1]{{\color#2[RGB]{138,88,84}#1}}
\newcommand<>\lightgray[1]{{\color#2[rgb]{0.8,0.8,0.8}#1}}
%\newcommand<>\highlightcolor[1]{{\color#2[rgb]{0,0,1}#1}}
\newcommand{\highlight}[1]{{\bf\steel{#1}}}
%\newcommand{\newblock}[0]{}

%\newcommand{\arrow}[0]{\includegraphics[height=5pt]{./figures/arrow}\hspace{3pt}}

\renewcommand{\emph}[1]{\textbf{\steel{{#1}}}}

\renewcommand{\alert}[1]{{\bf\red{{#1}}}}

\newcommand{\arrow}{
\begin{tikzpicture}
\draw [black!40!green, fill=black!40!green] (0,-0.12) -- (0,0.12) --
(0.15,0);
\draw [black!40!green, fill=black!40!green] (0.15,-0.12) -- (0.15,0.12) --
(0.3,0); 
\end{tikzpicture}
}

\geometry{left=0.45cm,top=0cm,right=0.45cm}


\newcommand{\logoimagepath}{./figures/imperial}
\newcommand{\highlightcolor}{blue!80!black}
%\newcommand{\headbarcolor}{imperialBlue}
\newcommand{\headbarcolor}{imperialBlue}
\institute{}

\newcommand{\coursetitle}{}

\newcommand{\slidesetsubtitle}{}
\newcommand{\slidesetnumber}{01}
\usefonttheme{professionalfonts}


\usetikzlibrary{decorations.fractals}
\input{../includes/tikzlibrarybayesnet.code.tex}
\input{../includes/tikzlibraryipe.code.tex}
\usetikzlibrary{matrix,positioning,decorations.pathreplacing}
\usetikzlibrary{calc,quotes,angles}
\usetikzlibrary{arrows, arrows.meta, patterns}

\usetikzlibrary{decorations.pathreplacing}
\tikzset{
    position label/.style={
       above = 3pt,
       text height = 2ex,
       text depth = 1ex
    }
}

% \usetikzlibrary{decorations.markings}
\tikzset{
  font={\fontsize{14pt}{12}\selectfont}
}



\useoutertheme[subsection=false,shadow]{miniframes}
\useinnertheme{default}
\usefonttheme{serif}
%\usepackage{palatino}
\usepackage{mathpazo}
%\usepackage{utopia}
\usepackage{stmaryrd} % for varodot, bigodot 
\usepackage{mathabx} % for \coAsterisk
%\usepackage{mnsymbol}
%\setbeamertemplate{itemize item}{\scriptsize\raise1.7pt\hbox{\donotcoloroutermaths$\Asterisk$}}
%\setbeamertemplate{itemize item}{\scriptsize\raise1.7pt\hbox{\donotcoloroutermaths$\varodot$}}
%\setbeamertemplate{itemize subitem}{\scriptsize\raise1.25pt\hbox{\donotcoloroutermaths$\rhd$}}

\usepackage{xifthen}% provides \isempty tesst

\setbeamerfont{title like}{shape=\scshape}
\setbeamerfont{frametitle}{}



\setbeamercolor*{lower separation line head}{bg=blue} 
\setbeamercolor*{normal text}{fg=black,bg=white} 
\setbeamercolor*{alerted text}{fg=red} 
\setbeamercolor*{example text}{fg=black} 
%\setbeamercolor*{frametitle}{fg=aimsbrown} 
\setbeamercolor*{frametitle}{fg=imperialBlue} 
\setbeamercolor*{structure}{fg=black} 
 
\setbeamercolor*{palette tertiary}{fg=black,bg=black!10} 
\setbeamercolor*{palette quaternary}{fg=black,bg=black!10} 

%\renewcommand{\(}{\begin{columns}}
%\renewcommand{\)}{\end{columns}}
%\newcommand{\<}[1]{\begin{column}{#1}}
%\renewcommand{\>}{\end{column}}

% ======================================
% custom commands 
\newcommand{\cemph}[1]{\textcolor{\highlightcolor}{#1}}
\newcommand{\calert}[1]{\textcolor{red}{#1}}

\setbeamertemplate{navigation symbols}{}
%\renewcommand\frametitle[1]{{\textsc{\Large \textcolor{\highlightcolor}{#1}}}\vspace{0.6cm}\par}

\setbeamertemplate{frametitle}
{
{\textsc\bf \insertframetitle}\vspace{0.2cm}\par
}


%%%%%%%%%%%%%%%%%%%%%%%%%%%%%%%%%%%%%%%%%%%%%%%%%%
\setbeamertemplate{headline}{% 
	\setbeamercolor{head1}{bg=\headbarcolor}
	 \hbox{%
  \begin{beamercolorbox}[wd=.01\paperwidth,ht=2.25ex,dp=50ex,center]{head1}%
  \fontsize{5}{5}\selectfont  
  \end{beamercolorbox}%
  }
  \vspace{-50ex}
}
\setbeamertemplate{footline}{
\begin{tiny}
\setbeamercolor{foot1}{fg=black,bg=gray!10}
\setbeamercolor{foot2}{fg=gray,bg=gray!15}
\setbeamercolor{foot3}{fg=gray,bg=gray!10}
\setbeamercolor{foot4}{fg=black,bg=gray!20}
\setbeamercolor{foot5}{fg=gray,bg=gray!15}
\setbeamercolor{foot6}{fg=black,bg=gray!20}

% taken from theme infolines and adapted
  \leavevmode%
  \hbox{%
  \begin{beamercolorbox}[wd=.45\paperwidth,ht=2.25ex,dp=1ex,center]{foot1}%
  \fontsize{5}{5}\selectfont
  \flushleft \hspace*{2ex}{\footertitle}
  \end{beamercolorbox}%
  % \begin{beamercolorbox}[wd=.08\paperwidth,ht=2.25ex,dp=1ex,center]{foot2}
  % \end{beamercolorbox}%
  %   \begin{beamercolorbox}[wd=.05\paperwidth,ht=2.25ex,dp=1ex,center]{foot3}
  % \end{beamercolorbox}%
    \begin{beamercolorbox}[wd=.45\paperwidth,ht=2.25ex,dp=1ex,center]{foot4}%
  \fontsize{5}{5}\selectfont
  \authorname\hspace{5mm}@\location, \talkDate%\ (\authorweb) 
  \end{beamercolorbox}%
  % \begin{beamercolorbox}[wd=.05\paperwidth,ht=2.25ex,dp=1ex,center]{foot5}
  % \end{beamercolorbox}%
  \begin{beamercolorbox}[wd=.1\paperwidth,ht=2.25ex,dp=1ex,right]{foot6}%
	\insertframenumber{}  \hspace*{2ex} 
  \end{beamercolorbox}}%
  \vskip0pt%
\end{tiny}
\vskip0pt
}


\setbeamertemplate{blocks}[rounded][shadow=false]


\newenvironment<>{myblock}[1]{%
  \begin{actionenv}#2%
      \def\insertblocktitle{#1}%
      \par%
      \mode<presentation>{%
%       \setbeamercolor{block title}{fg=black,bg=aimslightbrown!50!white}
      \setbeamercolor{block title}{fg=black,bg=imperialBlue!45!white}
       \setbeamercolor{block body}{fg=black,bg=gray!20}
       \setbeamercolor{itemize item}{fg=blue!40!white}
       \setbeamertemplate{itemize item}[triangle]
     }%
      \usebeamertemplate{block begin}}
    {\par\usebeamertemplate{block end}\end{actionenv}}

\newenvironment<>{myblock2}[1]{%
  \begin{actionenv}#2%
      \def\insertblocktitle{#1}%
      \par%
      \mode<presentation>{%
       \setbeamercolor{block title}{fg=white,bg=blue!80!black}
       \setbeamercolor{block body}{fg=black,bg=gray!20}
       \setbeamercolor{itemize item}{fg=green!60!black}
       \setbeamertemplate{itemize item}[triangle]
     }%
      \usebeamertemplate{block begin}}
    {\par\usebeamertemplate{block end}\end{actionenv}}

\gdef\colchar#1#2{%
  \tikz[baseline]{%
%  \node[anchor=base,inner sep=2pt,outer sep=0pt,fill = #2!20]
%  {\large{#1}};
  \node[anchor=base,inner sep=1pt,outer sep=0pt,fill = #2!20]
  {{\fontsize{11}{13}\selectfont #1}};
    }%
}%
\gdef\drawfontframe#1#2{%
  \tikz[baseline]{%
  \node[anchor=base,inner sep=2pt,outer sep=0pt,fill = #2!20] {#1};
    }%
  }%


\makeatletter
\let\@@magyar@captionfix\relax
\makeatother

%%% Local Variables:
%%% mode: latex
%%% TeX-master: "2018-09-arusha-linear-regression"
%%% End:



\input{../includes/titlepage.tex}
\linespread{1.2} 

\makeatletter
\newcommand{\Pause}[1][]{\unless\ifmeasuring@\relax
\pause[#1]%
\fi}
\makeatother







\section{Index Notation}

\begin{frame}{Vector Differentiation: Index Notation}
Back to our linear regression problem:
\begin{equation}
L(\vtheta) = \sum_{n=1}^N (y_n - \vphi(x_n)\transpose\vtheta)^2 = ||\vy - \Phi(X)\vtheta||^2
\end{equation}

How to find $\pderiv[L]{\vtheta} = [\pderiv[L]{\theta_i}]_i$? \pause

\begin{itemize}
\item Vector expressions can always be written in terms of operations on scalars. \pause
\item We know how to find partial derivatives of scalar expressions $\implies$ rewrite into index notation. \pause
\item Take partial derivatives using tricks.
\end{itemize}
\end{frame}


\begin{frame}{Index Notation of Vector Expressions}
\begin{align}
\norm{\vx}^2 &= \sum_i x_i^2 \\ \Pause
\vx\transpose\vy &= \sum_i x_i y_i \\ \Pause
\vy = f(\vx) \vx &\implies y_i = f(\vx) x_i \\ \Pause
\mat A \vx &=\, ? \\ \Pause
{[\mat{A} \vx]_i} &= \sum_{j} A_{ij} x_j \\ \Pause
\mat A\vx &= \left[\sum_j A_{ij} x_j\right]_i
\end{align}
\end{frame}


\begin{frame}{Index Notation of Vector Expressions}
Back to our linear regression problem:
\begin{align}
L(\vtheta) &= \sum_{n=1}^N (y_n - \vphi(x_n)\transpose\vtheta)^2 = ||\vy - \Phi(X)\vtheta||^2 \\
&= \sum_{n=1}^N \left(y_n - \sum_{m=1}^M\phi_m(x_n) \theta_m\right)^2
\end{align}

We want to find $\pderiv[L]{\vtheta}$, i.e.~$[\pderiv[L]{\theta_i}]_i$ for all $i$.\pause
\begin{align}
&\pderiv{\theta_i} \sum_{n=1}^N \left(y_n - \sum_{m=1}^M\phi_m(x_n) \theta_m\right)^2 \\ \Pause
 &= \sum_n 2\left(y_n - \sum_{m=1}^M\phi_m(x_n)\theta_m\right) \pderiv{\theta_i} \left(y_n - \sum_{k=1}^M\phi_k(x_n) \theta_k\right)
\end{align}
\end{frame}

\begin{frame}{Index Differentiation}
\begin{align}
\pderiv[L]{\theta_i} &=  -2\sum_{n} \left(y_n - \sum_{m=1}^M\phi_m(x_n)\theta_m\right) \sum_k \phi_k(x_n) \pderiv[\theta_k]{\theta_i}
\end{align} \pause
\begin{itemize}
\item Think about $\pderiv[\theta_k]{\theta_i}$. What value does it take?\pause
\item If $i\neq k$, then $0$; if $i = k$, then $1$. \pause
\item Denoted as \emph{kronecker delta} $\delta_{ik}$. \pause
\item Think about what happens to $w_i = \sum_k \delta_{ik}$. \pause
\item Think about what happens to $x_k = \sum_a v_a \delta_{ak}$. \pause Is $\delta_{ak}$ a matrix? \pause
\end{itemize}
\begin{align}
\pderiv[L]{\theta_i}  &= -2\sum_{n} \left(y_n - \sum_{m=1}^M\phi_m(x_n)\theta_m\right) \sum_k \phi_k(x_n) \delta_{ki} \\ \Pause
&= -2\sum_{n} \left(y_n - \sum_{m=1}^M\phi_m(x_n)\theta_m\right)\phi_i(x_n)
\end{align}
\end{frame}


\begin{frame}{Vector Notation of Index Expressions}
\begin{align}
\pderiv[L]{\theta_i}  &= -2\sum_{n} \left(y_n - \sum_{m=1}^M\phi_m(x_n)\theta_m\right)\phi_i(x_n)
\end{align}

Sometimes we prefer a compact vector expression again. Index notation is ambiguous about row/col vector. If we remember convention, we can go back:
\begin{align}
&\pderiv[L]{\vtheta} = \uncover<2->{2\sum_n (\vphi(x_n)\transpose\vtheta-y_n)\vphi(x_n)\transpose && \text{Remember: } \vphi(x_n) \in \Reals^M} \nonumber \\
&= \onslide<3->{2\sum_n [\Phi(X)\vtheta - \vy]_n \vphi(x_n)\transpose && \text{Define: } \Phi(X)\!\in\!\Reals^{N\times M}, \Phi_{n:}(X) = \vphi(\vx_n)\transpose} \nonumber \\
&= \onslide<4->{2(\Phi(X)\vtheta - \vy)\transpose \Phi(X)} \nonumber\\
&= \onslide<5->{2(\vtheta\transpose\Phi(X)\transpose\Phi(X) - \vy\transpose\Phi(X))}
\end{align}
\end{frame}


\section{Differentiation of vector-valued functions}

\begin{frame}
  \frametitle{Chain Rule}
Back to our linear regression problem:
\begin{equation}
L(\vtheta) = \sum_{n=1}^N (y_n - \vphi(x_n)\transpose\vtheta)^2 = ||\vy - \Phi(X)\vtheta||^2
\end{equation}

\pause
\begin{itemize}
\item We could manually take partial derivatives of $L(\vtheta)$ \pause
\item Or, we could see our scalar function $L(\vtheta)$ as a function composition:
\end{itemize}
\begin{align*}
f(\vec g(\vtheta)) && f: \Reals^{D}\to\Reals \,, \qquad \vec g: \Reals^{E}\to\Reals^D
\end{align*}
\pause
\vspace{-0.4cm}
\begin{itemize}
\item Wouldn't it be nice to have a chain rule? \pause $\diffF{f(\vec g(\vtheta))}{\vtheta} = \diffF{f}{\vec g}\diffF{\vec g}{\vtheta}$ \pause
\item But how does a derivative of a vector w.r.t.~a vector work?
\end{itemize}
\end{frame}


\begin{frame}{Differentiation of a vector w.r.t.~a scalar}
\begin{itemize}
\item Differentiation quantifies how an output changes, \\
\hspace{4.3cm} in response to an input change. \pause
\item Change is measured by subtraction. {\color{gray} Draw vector subtraction.} \pause
\item Can use the same limiting argument as with scalars: \pause
\begin{align}
\deriv[\vx]{t} = \lim_{\Delta t\to 0} \frac{\vx(t + \Delta t) - \vx(t)}{\Delta t}
\end{align}
But now, derivative is a vector! \pause
\item Subtraction is elementwise, so we can reduce to scalar case:
\begin{align}
\left[\deriv[\vx]{t}\right]_i = \lim_{\Delta t\to 0} \frac{x_i(t + \Delta t) - x_i(t)}{\Delta t}
\end{align}
\end{itemize}
\end{frame}


\begin{frame}{Example: Circle}
Function that describes a point going round a circle, with period 1s:
\begin{align}
\vx = \begin{bmatrix}\cos 2\pi t & \sin 2\pi t\end{bmatrix}\transpose
\end{align}
{\color{gray} Draw on board} \pause

\begin{itemize}
\item Based on intuition, what should the speed be? \pause $2\pi$ m/s. \pause
\item Find velocity vector $\calcd\vx / \calcd t$.\pause
\item Find speed from the norm.
\end{itemize}
\end{frame}


\begin{frame}{Summary: Differentiation of a vector w.r.t.~a scalar}
\begin{myblock}{Derivative of a vector}
The derivative of a vector-valued function is given by the derivative of each of its elements.
\end{myblock}
\end{frame}







\section{Multivariate Chain Rule}
\begin{frame}[t]{Multivariate Chain Rule w.r.t.~scalar}
It turns out, there is a multivariate chain rule:
\only<1>{
\begin{align*}
\deriv[f(a(t), b(t))]{t} = \pderiv[f]{a}\deriv[a]{t} + \pderiv[f]{b}\deriv[b]{t}
\end{align*}}
\only<2->{
\begin{align*}
\deriv[f(\vec g(t))]{t} = \sum_{i=1}^D\pderiv[f]{g_i}\deriv[g_i]{t} && \vec g(t) \in \Reals^D
\end{align*}} \pause \pause
Inner product!  \pause \emph{Given our convention}, we can write in vector form:
\begin{align*}
\deriv[f(\vec g(t))]{t} = \underbrace{\deriv[f]{\vec g}}_{\text{row}}\cdot\underbrace{\deriv[\vec g]{t}}_{\text{column}}
\end{align*} \pause
\vspace{-0.3cm}
\begin{itemize}
\item $\deriv[\vec g]{t}$ is the derivative of a column vector. We keep this to be a column vector.
\item Can also be derived from a limit argument, like the scalar derivative (\emph{board}: Circle Example).
\end{itemize}
\end{frame}


\begin{frame}
  \frametitle{Example: Chain Rule}
  \begin{itemize}
    \item 
  Consider $f:\R^2\to \R, \quad \vec x:\R \to \R^2$
  \begin{align*}
    f(\vec x)&= f(x_1, x_2) = x_1^2+2x_2\,,\\
    \vec x(t) &=
                \begin{bmatrix}
                  x_1(t)\\ x_2(t)
                \end{bmatrix} =
                \begin{bmatrix}
      \sin(t)\\ \cos (t)
    \end{bmatrix}
  \end{align*}
  \vspace{-2mm}
  \pause
\item What are the dimensions of \blue{$\diffF{f}{\vec x}$} and \red{$\diffF{ \vec x}{t}$}?\\
  \only<2>{\qquad \qquad \qquad \emph{Work it out with your neighbors}}
  \pause \qquad \qquad \qquad \qquad \qquad \blue{$1\times 2$} and \red{$2\times 1$}
\item Compute the gradient $\diffF{f}{t}$ using the chain rule:
  \pause
  \begin{align*}
    \diffF{ f}{t} = \blue{\diffF{f}{\vec x}} \red{\diffF{\vec x}{t}}
     &= \blue{\begin{bmatrix}
		\frac{\partial f}{\partial x_1} & 
		\frac{\partial f}{\partial x_2}
		\end{bmatrix}}
		\red{\begin{bmatrix}
    	\frac{\partial x_1}{\partial t}\\[2mm]
        \frac{\partial x_2}{\partial t}
     	\end{bmatrix}}
 	= \blue{\begin{bmatrix}
		2\sin t & 2
		\end{bmatrix}}
        \red{\begin{bmatrix}
        \cos t\\
        -\sin t
        \end{bmatrix}}\\
     &= 2\sin t\cos t - 2\sin t = 2\sin t(\cos t -1)
  \end{align*}
\end{itemize}
\end{frame}




\begin{frame}[t]{Derivative of vector w.r.t.~vector}
We saw the chain rule if we were differentiating w.r.t.~a scalar:
\only<1>{
\begin{align*}
\deriv[f(a(t), b(t))]{t} = \pderiv[f]{a}\deriv[a]{t} + \pderiv[f]{b}\deriv[b]{t}
\end{align*}}
\only<2->{
\begin{align*}
\deriv[f(\vec g(t))]{t} = \sum_{i=1}^D\pderiv[f]{g_i}\deriv[g_i]{t} = \underbrace{\deriv[f]{\vec g}}_{\text{row}}\cdot\underbrace{\deriv[\vec g]{t}}_{\text{column}} && \vec g(t) \in \Reals^D
\end{align*}} \pause \pause
What happens if we differentiate w.r.t.~a vector? \pause \\
$\implies$ As before, we just stack the derivatives w.r.t.~each of the inputs.
\begin{align*}
\pderiv[f(\vec g(\vx))]{x_j} = \sum_{i=1}^D\pderiv[f]{g_i}\pderiv[g_i]{x_j} && \vec g(\vx) \in \Reals^D
\end{align*}
\end{frame}





\begin{frame}{Multivariate Chain Rule w.r.t.~vector}
This is a matrix multiplication! Can write in vector form:
\begin{align*}
\deriv[f(\vec g(\vx))]{\vx} = \underbrace{\deriv[f]{\vec g}}_{\text{row}}\cdot\underbrace{\deriv[\vec g]{\vx}}_{\text{matrix}}
\end{align*} \pause
\begin{itemize}
\item $\deriv[\vec g]{\vx}$ is the derivative of a column vector w.r.t.~the input vector $\vx$. We put the elements of $\vec g$ (i.e.~i) along the column, and the dimensions of the derivative (i.e.~j) along the row. \pause
\item Can also be derived from a directional derivative argument, but for a vector.
\end{itemize}
\end{frame}



\begin{frame}
  \frametitle{Vector Field Differentiation $f: \R^N\to\R^M$}
  \begin{align*}
    \vec y &= \vec f(\vec x) \in\R^M\,,\quad \vec x \in \R^N\\
    \begin{bmatrix}
     \colchar{$ y_1$}{red}\\
     \vdots\\
     \colchar{$y_M$}{blue}
    \end{bmatrix}
    &=
    \begin{bmatrix}
      \colchar{$f_1(\vec x)$}{red}\\
      \vdots\\
      \colchar{$f_M(\vec x)$}{blue}
    \end{bmatrix}
    =
    \begin{bmatrix}
      \colchar{$f_1(x_1,\dotsc, x_N)$}{red}\\
      \vdots\\
      \colchar{$f_M(x_1,\dotsc, x_N)$}{blue}
    \end{bmatrix}
  \end{align*}
  
  \pause
  \begin{itemize}    
    \item \cemph{Jacobian} matrix (collection of all partial derivatives)
      \begin{align*}
        \begin{bmatrix}
          \colchar{$\frac{d y_1}{d\vec x}$}{red}\\
          \vdots\\
          \colchar{$\frac{d y_M}{d\vec x}$}{blue}
        \end{bmatrix}
        =
        \begin{bmatrix}
          \colchar{$\frac{\partial f_1}{\partial
              x_1}$}{red} &
          \cdots & \colchar{$\frac{\partial
              f_1}{\partial x_N}$}{red}\\
          \vdots &  & \vdots\\
          \colchar{$\frac{\partial f_M}{\partial x_1}$}{blue} & \cdots & \colchar{$\frac{\partial
            f_M}{\partial x_N}$}{blue}
        \end{bmatrix}
                                                       \in\R^{M\times N}
      \end{align*}
    \end{itemize}
  
\end{frame}





\begin{frame}{Dimensionality of the Gradient}
  In general: A function $\vec
  f:\R^\text{\colchar{\scriptsize{$N$}}{blue}}\to\R^\text{\colchar{\scriptsize{$M$}}{orange}}$
  has a gradient that is an
  $M\times N$-matrix
  with
    $$
    \diffF{\vec f}{\vec x}\in \R^{M\times N}\,,\qquad \mathrm{d}\vec f[m,n] =
    \frac{\partial f_m}{\partial x_n}
    $$
    \begin{center}
     Gradient dimension: \colchar{ \# target dimensions }{orange} $\times$ \colchar{\# input dimensions}{blue}
    \end{center}

\pause
A function composition $\vf(\vx) = \vf_{\vg}(\vg(\vx))$ {\tiny(subscript to distinguish from overall function)} has the constraint that the \emph{output dimension} of $\mathbf g(\cdot)$ has to equal the \emph{input dimension} of $\vf_{\vg}(\cdot)$, so that we can compute $\vf(\mathbf g(\vx))$. \pause

This ensures that the shapes of the chain rule work out:
\begin{gather}
\vf : \Reals^\text{\colchar{\scriptsize{$N$}}{blue}} \to \Reals^\text{\colchar{\scriptsize{$M$}}{orange}} \qquad \qquad \vf_{\vg} : \Reals^\text{\colchar{\scriptsize{$L$}}{green}} \to \Reals^\text{\colchar{\scriptsize{$M$}}{orange}} \qquad \qquad \mathbf g : \Reals^\text{\colchar{\scriptsize{$N$}}{blue}} \to \Reals^\text{\colchar{\scriptsize{$L$}}{green}} \\
\underbrace{\diffF{\vf}{\vx}}_{\text{\colchar{\scriptsize{$M$}}{orange}}\times\text{\colchar{\scriptsize{$N$}}{blue}}} = \underbrace{\diffF{\vf_{\vg}}{\mathbf g}}_{\text{\colchar{\scriptsize{$M$}}{orange}}\times\text{\colchar{\scriptsize{$L$}}{green}}} \underbrace{\diffF{\mathbf g}{\vx}}_{\text{\colchar{\scriptsize{$L$}}{green}}\times\text{\colchar{\scriptsize{$N$}}{blue}}}
\end{gather}

    
\end{frame}




\begin{frame}
\frametitle{Example: Vector Field Differentiation}

\vspace{-5mm}
\begin{align*}
\vec f(\vec x) = \mat A\vec x\,,\qquad \vec f(\vec x)\in\R^M, \quad\mat A\in\R^{M\times
  N}, \quad \vec x\in\R^N
\end{align*}

\vspace{-5mm}
{\scriptsize 
		\begin{align*}
        \begin{bmatrix}
          \colchar{$y_1$}{red}\\
          \vdots\\
          \colchar{$y_M$}{blue}
        \end{bmatrix}
        =
        \begin{bmatrix}
          \colchar{$f_1(\vec x)$}{red}\\
          \vdots\\
          \colchar{$f_M(\vec x)$}{blue}
        \end{bmatrix}
        =
        \begin{bmatrix}
          \colchar{$A_{11}x_1$}{red} + \colchar{$A_{12}x_2$}{red} + &
          \cdots & + \colchar{$A_{1N}x_N$}{red}\\
          \vdots \qquad \qquad \vdots & \vdots & \quad  \vdots \\
          \colchar{$A_{M1}x_1$}{blue} + \colchar{$A_{M2}x_2$}{blue} + & 
          \cdots & 
          + \colchar{$A_{MN}x_N$}{blue}
        \end{bmatrix}
      \end{align*}}
\vspace{-1mm}
\begin{itemize}
\item Compute the gradient $\diffF{\vec f}{\vec x}$
  \begin{itemize}
  %\item Dimension of $\diffF{\vec f}{\vec x}$:
  %  \pause Since $\vec f:\R^N\to \R^M$ it follows that
  %  $\diffF{\vec f}{\vec x}\in\R^{M\times N}$
    \pause
  \item Gradient:
  	\vspace{-1mm}
    \begin{align*}
      f_{i}(\vec x) &= \sum_{k=1}^NA_{ik}x_k \qquad 
      \implies \frac{\partial f_i}{\partial x_j} =
              \sum_{k}A_{ik}\pderiv[x_k]{x_j} = \sum_{k}A_{ik}\delta_{kj} = A_{ij} \nonumber\\
      \onslide+<4>{
      \implies \diffF{\vec f}{\vec x} &= \begin{bmatrix}
        \frac{\partial f_1}{\partial x_1} & \cdots & \frac{\partial
          f_1}{\partial x_N}\\
        \vdots & & \vdots\\
        \frac{\partial f_M}{\partial x_1} & \cdots & \frac{\partial
          f_M}{\partial x_N}
      \end{bmatrix}
                                                     =
                                                     \begin{bmatrix}
A_{11} & \cdots & A_{1N}\\
\vdots & &\vdots\\
A_{M1} & \cdots & A_{MN}
\end{bmatrix}
                  =\mat A \in\R^{M\times N}}
    \end{align*}
  \end{itemize}
\end{itemize}
\end{frame}




%%%%%%%%%%%%%%%%%%%%%%%%%%%%%%%%%%%%%%%%%
\begin{frame}
  \frametitle{Example: Multivariate Chain Rule}
  \begin{itemize}
    \item Consider the function
  \begin{align*}
    L(\vec e) &= \tfrac{1}{2}\|\vec e\|^2 = \tfrac{1}{2}\vec e\T\vec e\\
    \vec e &= \vec y - \mat A\vec x\,,\quad \vec x\in\R^N\,,\mat
             A\in\R^{M\times N}\,,\vec e,\vec y\in\R^M
  \end{align*}
\item Compute the gradient $\diffF{L}{\vec x}$. What is the dimension/size of $\diffF{L}{\vec
  x}$?
  \only<1>{
  	\vspace{2mm}
	\begin{center}
	\textbf{Work it out with your neighbours}
	\end{center}}

  \pause
  \begin{align}
\diffF{L}{\vec x} &= \blue{\pdiffF{L}{\vec
                     e}}\red{\pdiffF{\vec e}{\vec
                     x}}\nonumber\\
    \blue{\pdiffF{L}{\vec e}} &\blue{=\vec e\T}\in
                                               \R^{1\times M}\,, \quad\quad \pderiv[L]{e_i} = \pderiv[]{e_i} \sum_{j}\frac{1}{2} e_j^2 = \sum_j \frac{1}{2} 2e_j \pderiv[e_j]{e_i} = e_i \nonumber \\
    \red{\pdiffF{\vec e}{\vec x}} &\red{=-\mat
                                                     A}\in\R^{M\times
                                                     N} \nonumber\\
    \implies \diffF{L}{\vec x} &=\blue{\vec e\T}(\red{-\mat A}) =
                                     \red{-}\blue{(\vec y - \mat
                                     A\vec x)\T} \red{\mat
                                     A}\in\R^{1\times N}\nonumber
  \end{align}
\end{itemize}
\end{frame}




%%%%%%%%%%%%%%%%%%%%%%%%%%%%%%%%%%%%%%%%%

\begin{frame}{Summary}
\begin{itemize}
\item Chain rule for multivariate functions
\item Derivatives of vectors w.r.t.~scalars.
\item Derivatives of vectors w.r.t.~vectors (and shapes).
\end{itemize}
\end{frame}





\end{document}
%%% Local Variables: 
%%% mode: latex
%%% TeX-master: t
%%% End: 
